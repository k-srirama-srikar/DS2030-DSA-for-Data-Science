\documentclass{article}

\title{\textbf{DS2030 - Data Structures and Algorithms for Data Science \\Lab 1 (Take Home)}}
\author{K Srirama Srikar \\142301013 }
\date{}

\begin{document}

\maketitle
\section{Question 1}
\subsection{Case - 1}
Open terminal and run the following command \\ \texttt{python3 Case1.py N x y} \\  Here \texttt{N} is the size of the grid, \texttt{x} \texttt{y} are the x and y coordinates from where the robots starts respectively.  \\
This should result in an integer output, \texttt{n} \\
Note that the file named \texttt{steps.py} is a part of \texttt{Case1.py} and \texttt{Case2.py} . It contains the recursive logic to calculate the number of steps. \\

\subsection{Case - 2}
Open terminal and run the following command \\ \texttt{python3 Case2.py N x y cx cy} \\  Here \texttt{N} is the size of the grid, \texttt{x} \texttt{y} are the x and y coordinates from where the robots starts respectively, and, \texttt{cx} \texttt{cy} are the x and y coordinates of the charging station respectively.\\ The resulting output is an integer \texttt{n} ... 

\end{document} 

